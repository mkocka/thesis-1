%%%%%%%%%%%%%%%%%%%%%%%%%%%%%%%%%%%%%%%%%%%%%%%%%%%%%%%%%%%%%%%%%%%%
%% I, the copyright holder of this work, release this work into the
%% public domain. This applies worldwide. In some countries this may
%% not be legally possible; if so: I grant anyone the right to use
%% this work for any purpose, without any conditions, unless such
%% conditions are required by law.
%%%%%%%%%%%%%%%%%%%%%%%%%%%%%%%%%%%%%%%%%%%%%%%%%%%%%%%%%%%%%%%%%%%%

\documentclass[
  color, %% This option enables colorful typesetting. Replace with
         %% `monochrome`, if you are going to print the thesis on
         %% a monochromatic printer.
  table, %% Causes the coloring of tables. Replace with `notable`
         %% to restore plain tables.
  lof,   %% Prints the List of Figures. Replace with `nolof` to
         %% hide the List of Figures.
  lot,   %% Prints the List of Tables. Replace with `nolot` to
         %% hide the List of Tables.
  %% More options are listed in the class documentation at
  %% <http://mirrors.ctan.org/macros/latex/contrib/fithesis/fithesis/guide/mu/fi.pdf>.
]{fithesis3}
%% The following section sets up the locales used in the thesis.
\usepackage[resetfonts]{cmap} %% We need to load the T2A font encoding
\usepackage[T1,T2A]{fontenc}  %% to use the Cyrillic fonts with Russian texts.

\usepackage[
  main=english, %% By using `czech` or `slovak` as the main locale
                %% instead of `english`, you can typeset the thesis
                %% in either Czech or Slovak, respectively.
  german, russian, czech, slovak %% The additional keys allow
]{babel}        %% foreign texts to be typeset as follows:
%%
%%   \begin{otherlanguage}{german}  ... \end{otherlanguage}
%%   \begin{otherlanguage}{russian} ... \end{otherlanguage}
%%   \begin{otherlanguage}{czech}   ... \end{otherlanguage}
%%   \begin{otherlanguage}{slovak}  ... \end{otherlanguage}
%%
%% For non-Latin scripts, it may be necessary to load additional
%% fonts:
\usepackage{paratype}
\def\textrussian#1{{\usefont{T2A}{PTSerif-TLF}{m}{rm}#1}}
%%
%% The following section sets up the metadata of the thesis.
\thesissetup{
    university    = mu,
    faculty       = fi,
    type          = bc,
    author        = Jane Doe,
    gender        = f,
    advisor       = John Smith,
    title         = {The Proof of P = NP},
    TeXtitle      = {The Proof of $\mathsf{P}=\mathsf{NP}$},
    keywords      = {keyword1, keyword2, ...},
    TeXkeywords   = {keyword1, keyword2, \ldots},
}
\thesislong{abstract}{
    This is the abstract of my thesis, which can

    span multiple paragraphs.
}
\thesislong{thanks}{
    This is the acknowledgement for my thesis, which can

    span multiple paragraphs.
}
%% The following section sets up the bibliography.
\usepackage{csquotes}
\usepackage[              %% When typesetting the bibliography, the
  backend=biber,          %% `numeric` style will be used for the
  style=numeric,          %% entries and the `numeric-comp` style
  citestyle=numeric-comp, %% for the references to the entries. The
  sorting=none,           %% entries will be sorted in cite order.
  sortlocale=auto         %% For more unformation about the available
]{biblatex}               %% `style`s and `citestyles`, see:
%% <http://mirrors.ctan.org/macros/latex/contrib/biblatex/doc/biblatex.pdf>.
\addbibresource{example.bib} %% The bibliograpic database within
                          %% the file `example.bib` will be used.
\usepackage{makeidx}      %% The `makeidx` package contains
\makeindex                %% helper commands for index typesetting.
%% These additional packages are used within the document:
\usepackage{paralist}
\usepackage{amsmath}
\usepackage{amsthm}
\usepackage{amsfonts}
\usepackage{url}
\usepackage{menukeys}
\begin{document}
\chapter{Introduction}
Theses are rumoured to be the capstones of education, so I decided
to write one of my own. If all goes well, I will soon have a
diploma under my belt. Wish me luck!

\begin{otherlanguage}{czech}
Říká se, že závěrečné práce jsou vyvrcholením studia a tak jsem se
rozhodl jednu také napsat. Pokud vše půjde podle plánu, odnesu si
na konci semestru diplom. Držte mi palce!
\end{otherlanguage}

\begin{otherlanguage}{slovak}
Hovorí sa, že záverečné práce sú vyvrcholením štúdia a tak som sa
rozhodol jednu tiež napísať. Ak všetko pôjde podľa plánu, odnesiem
si na konci semestra diplom. Držte mi palce!
\end{otherlanguage}

\begin{otherlanguage}{german}
Man munkelt, dass die Dissertation die Krönung der Ausbildung ist.
Deshalb habe ich mich beschlossen meine eigene zu schreiben. Wenn
alles gut geht, bekomme ich bald ein Diplom. Wünsch mir Glück!
\end{otherlanguage}

\begin{otherlanguage}{russian}\textrussian{%
Говорят, что тезис -- это кульминация обучения. Поэтому я и решил
написать собственный тезис. Если всё сработает по плану, я скоро
получу диплом. Желайте мне удачи!
}\end{otherlanguage}

\chapter{These are}
\section{the available}
\subsection{sectioning commands.}
\paragraph{Paragraphs and}
\subparagraph{subparagraphs are available as well.}
Inside the text, you can also use unnumbered lists,
\begin{itemize}
  \item such as
  \item this one
  \begin{itemize}
    \item     and they can be nested as well.
    \item[>>] You can even turn the bullets into something fancier,
    \item[\S] if you so desire.
  \end{itemize}
\end{itemize}
Numbered lists are
\begin{enumerate}
  \item very
  \begin{enumerate}
    \item similar
  \end{enumerate}
\end{enumerate}
and so are description lists:
\begin{description}
  \item[Description list]
    A list of terms with a description of each term
\end{description}
The spacing of these lists is geared towards paragraphs of text.
For lists of words and phrases, the \textsf{paralist} package
offers commands
\begin{compactitem}
  \item that
  \begin{compactitem}
    \item are
    \begin{compactitem}
      \item better
      \begin{compactitem}
        \item suited
      \end{compactitem}
    \end{compactitem}
  \end{compactitem}
\end{compactitem}
\begin{compactenum}
  \item to
  \begin{compactenum}
    \item this
    \begin{compactenum}
      \item kind of
      \begin{compactenum}
        \item content.
      \end{compactenum}
    \end{compactenum}
  \end{compactenum}
\end{compactenum}
The \textsf{amsthm} package provides the commands necessary for the
typesetting of mathematical definitions, theorems, lemmas and
proofs.

%% We will define several mathematical sectioning commands.
\newtheorem{theorem}{Theorem}[section] %% The numbering of theorems
                               %% will be reset after each section.
\newtheorem{lemma}[theorem]{Lemma}     %% The numbering of lemmas
\newtheorem{corr}[theorem]{Corrolary}  %% and corrolaries will
                                %% share the counter with theorems.
\theoremstyle{definition}
\newtheorem{definition}{Definition}
\theoremstyle{remark}
\newtheorem*{remark}{Remark}

\begin{theorem}
  This is a theorem that offers a profound insight into the
  mathematical sectioning commands.
\end{theorem}
\begin{theorem}[Another theorem]
  This is another theorem. Unlike the first one, this theorem has
  been endowed with a name.
\end{theorem}
\begin{lemma}
  Let us suppose that $x^2+y^2=z^2$. Then
  \begin{equation}
    \biggl\langle u\biggm|\sum_{i=1}^nF(e_i,v)e_i\biggr\rangle
    =F\biggl(\sum_{i=1}^n\langle e_i|u\rangle e_i,v\biggr).
  \end{equation}
\end{lemma}
\begin{proof}
  $\nabla^2 f(x,y)=\frac{\partial^2f}{\partial x^2}+
   \frac{\partial^2f}{\partial y^2}$.
\end{proof}
\begin{corr}
  This is a corrolary.
\end{corr}
\begin{remark}
  This is a remark.
\end{remark}

\chapter{Floats and references}
\begin{figure}
  \begin{center}
    %% PNG and JPG images can be inserted into the document as well,
    %% but their resolution needs to be adequate. The minimum is
    %% about 250 pixels per 1 centimeter. That means that a JPG or
    %% PNG image typeset at 40 × 40 mm should be 1000 × 1000 px
    %% large at minimum.
    \includegraphics[width=40mm]{fithesis/logo/mu/fithesis-base.pdf}
  \end{center}
  \caption{The logo of the Masaryk University at 40\,mm}
  \label{fig:mulogo1}
\end{figure}

\begin{figure}
  \begin{minipage}{.66\textwidth}
    \includegraphics[width=\textwidth]{fithesis/logo/mu/fithesis-base.pdf}
  \end{minipage}
  \begin{minipage}{.33\textwidth}
    \includegraphics[width=\textwidth]{fithesis/logo/mu/fithesis-base.pdf} \\
    \includegraphics[width=\textwidth]{fithesis/logo/mu/fithesis-base.pdf}
  \end{minipage}
  \caption{The logo of the Masaryk University at $\frac23$ and
    $\frac13$ of text width}
  \label{fig:mulogo2}
\end{figure}

\begin{table}
  \begin{tabularx}{\textwidth}{lllX}
    \toprule
    Day & Min Temp & Max Temp & Summary \\
    \midrule
    Monday & $13^{\circ}\mathrm{C}$ & $21^\circ\mathrm{C}$ & A
    clear day with low wind and no adverse current advisories. \\
    Tuesday & $11^{\circ}\mathrm{C}$ & $17^\circ\mathrm{C}$ & A
    trough of low pressure will come from the northwest. \\
    Wednesday & $10^{\circ}\mathrm{C}$ &
    $21^\circ\mathrm{C}$ & Rain will spread to all parts during the
    morning. \\
    \bottomrule
  \end{tabularx}
  \caption{A weather forecast}
  \label{tab:weather}
\end{table}

The logo of the Masaryk University is shown in Figure
\ref{fig:mulogo1} and Figure \ref{fig:mulogo2} at pages
\pageref{fig:mulogo1} and \pageref{fig:mulogo2}. The weather
forecast is shown in Table \ref{tab:weather} at page
\pageref{tab:weather}. The following chapter is Chapter
\ref{chap:matheq} and starts at page \pageref{chap:matheq}.
Items \ref{item:star1}, \ref{item:star2}, and
\ref{item:star3} are starred in the following list:
\begin{compactenum}
  \item some text
  \item some other text
  \item $\star$ \label{item:star1}
  \begin{compactenum}
    \item some text
    \item $\star$ \label{item:star2}
    \item some other text
    \begin{compactenum}
      \item some text
      \item some other text
      \item yet another piece of text
      \item $\star$ \label{item:star3}
    \end{compactenum}
    \item yet another piece of text
  \end{compactenum}
  \item yet another piece of text
\end{compactenum}
If your reference points to a place that has not yet been typeset,
the \verb"\ref" command will expand to \textbf{??} during the first
run of
\texttt{pdflatex \jobname.tex}
and a second run is going to be needed for the references to
resolve. With online services -- such as Overleaf -- this is
performed automatically.

\chapter{Mathematical equations}
\label{chap:matheq}
\TeX{} comes pre-packed with the ability to typeset inline
equations, such as $\mathrm{e}^{ix}=\cos x+i\sin x$, and display
equations, such as \[
  \mathbf{A}^{-1} = \begin{bmatrix}
  a & b \\ c & d \\
  \end{bmatrix}^{-1} =
  \frac{1}{\det(\mathbf{A})} \begin{bmatrix}
  \,\,\,d & \!\!-b \\ -c & \,a \\
  \end{bmatrix} =
  \frac{1}{ad - bc} \begin{bmatrix}
  \,\,\,d & \!\!-b \\ -c & \,a \\
  \end{bmatrix}.
\] \LaTeX{} defines the automatically numbered \texttt{equation}
environment:
\begin{equation}
  \gamma Px = PAx = PAP^{-1}Px.
\end{equation}
The package \textsf{amsmath} provides several additional
environments that can be used to typeset complex equations:
\begin{enumerate}
  \item An equation can be spread over multiple lines using the
    \texttt{multline} environment:
    \begin{multline}
      a + b + c + d + e + f + b + c + d + e + f + b + c + d + e +
f \\
      + f + g + h + i + j + k + l + m + n + o + p + q
    \end{multline}

  \item Several aligned equations can be typeset using the
    \texttt{align} environment:
    \begin{align}
              a + b &= c + d     \\
                  u &= v + w + x \\[1ex]
      i + j + k + l &= m
    \end{align}

  \item The \texttt{alignat} environment is similar to
    \texttt{align}, but it doesn't insert horizontal spaces between
    the individual columns:
    \begin{alignat}{2}
      a + b + c &+ d       &   &= 0 \\
              e &+ f + g   &   &= 5
    \end{alignat}

  \item Much like chapter, sections, tables, figures, or list
    items, equations -- such as \eqref{eq:first} and
    \eqref{eq:mine} -- can also be labeled and referenced:
    \begin{alignat}{4}
      b_{11}x_1 &+ b_{12}x_2  &  &+ b_{13}x_3  &  &             &
        &= y_1,                   \label{eq:first} \\
      b_{21}x_1 &+ b_{22}x_2  &  &             &  &+ b_{24}x_4  &
        &= y_2. \tag{My equation} \label{eq:mine}
    \end{alignat}

  \item The \texttt{gather} environment makes it possible to
    typeset several equations without any alignment:
    \begin{gather}
      \psi = \psi\psi, \\
      \eta = \eta\eta\eta\eta\eta\eta, \\
      \theta = \theta.
    \end{gather}

  \item Several cases can be typeset using the \texttt{cases}
    environment:
    \begin{equation}
      |y| = \begin{cases}
        \phantom-y & \text{if }z\geq0, \\
                -y & \text{otherwise}.
      \end{cases}
    \end{equation}
\end{enumerate}
For the complete list of environments and commands, consult the
\textsf{amsmath} package manual\footnote{
  See \url{http://mirrors.ctan.org/macros/latex/required/amslatex/math/amsldoc.pdf}.
  The \texttt{\textbackslash url} command is provided by the
  package \textsf{url}.
}.

\chapter{\textnormal{We \textsf{have} \texttt{several} \textsc{fonts}
  \textit{at} \textbf{disposal}}}
The serified roman font is used for the main body of the text.
\textit{Italics are typically used to denote emphasis or
quotations.} \texttt{The teletype font is typically used for source
code listings.} The \textbf{bold}, \textsc{small-caps} and
\textsf{sans-serif} variants of the base roman font can be used to
denote specific types of information.

\tiny We \scriptsize can \footnotesize also \small change \normalsize
the \large font \Large size, \LARGE although \huge it \Huge
is \huge usually \LARGE not \Large necessary.\normalsize

A wide variety of mathematical fonts is also available, such as: \[
  \mathrm{ABC}, \mathcal{ABC}, \mathbf{ABC}, \mathsf{ABC},
  \mathit{ABC}, \mathtt{ABC}
\] By loading the \textsf{amsfonts} packages, several additional
fonts will become available: \[
  \mathfrak{ABC}, \mathbb{ABC}
\] Many other mathematical fonts are available\footnote{
  See \url{http://tex.stackexchange.com/a/58124/70941}.
}.

\chapter{Inserting the bibliography}
After loading the \texttt{biblatex} package and linking a
bibliography data\-base file to the document using the
\verb"\addbibresource" command, you can start citing the entries.
This is just dummy text \cite{inbook-full} lightly sprinkled with
citations \cite[p.~123]{incollection-full}.  Several sources can be
cited at once \cite{whole-collection, manual-minimal,manual-full}.
\citetitle{inbook-full} was written by \citeauthor{inbook-full} in
\citeyear{inbook-full}. We can also produce \textcite{inbook-full}
or%% Let us define a compound command:
\def\citeauthoryear#1{(\textcite{#1},~\citeyear{#1})}
\citeauthoryear{inbook-full}. The full bibliographic citation is:
\emph{\fullcite{inbook-full}}. We can easily insert a bibliographic
citation into the footnote\footfullcite{inbook-full}.

The \verb"\nocite" command will not generate any
output\nocite{booklet-full}, but it will insert its argument into
the bibliography. The \verb"\nocite{*}" command will insert all the
records in the bibliography database file into the bibliography.
Try uncommenting the command
%% \nocite{*}
and watch the bibliography section come apart at the seams.

When typesetting the document for the first time, citing a
\texttt{work} will expand to [\textbf{work}] and the
\verb"\printbibliography" command will produce no output. It is now
necessary to generate the bibliography by running \texttt{biber
\jobname.bcf} from the command line and then by typesetting the
document again twice. During the first run, the bibliography
section and the citations will be typeset, and in the second run,
the bibliography section will appear in the table of contents.

The \texttt{biber} command needs to be executed from within the
directory, where the \LaTeX\ source file is located. In Windows,
the command line can be opened in a directory by holding down the
\keys{Shift} key and by clicking the right mouse button while
hovering the cursor over a directory.  Select the \menu{Open
Command Window Here} option in the context menu that opens shortly
afterwards.

With online services -- such as Overleaf -- all commands are
executed automatically.

{\csname captions\languagename\endcsname %% Temporarily override
%% the BibLaTeX localization with the original babel definitions.
\makeatletter %% Use the correct localization of the quotations.
  \thesis@selectLocale{\thesis@locale}\makeatother
\printbibliography[heading=bibintoc]} %% Print the bibliography.

\chapter{Inserting the index}
After using the \verb"\makeindex" macro and loading the
\texttt{makeidx} package that provides additional indexing
commands, index entries can be created by issuing the \verb"\index"
command. \index{dummy text|(}It is possible to create ranged index
entries, which will encompass a span of text.\index{dummy text|)}
To insert complex typographic material -- such as $\alpha$
\index{alpha@$\alpha$} or \TeX{} \index{TeX@\TeX} --
into the index, you need to specify a text string, which will
determine how the entry will be sorted. It is also possible to
create hierarchal entries. \index{vehicles!trucks}
\index{vehicles!speed cars}

After typesetting the document, it is necessary to generate the
index by running
\begin{center}%
  \texttt{texindy -I latex -C utf8 -L }$\langle$\textit{locale}%
  $\rangle$\texttt{ \jobname.idx}
\end{center}
from the command line, where $\langle$\textit{locale}$\rangle$
corresponds to the main locale of your thesis -- such as
\texttt{english}, and then typesetting the document again.

The \texttt{texindy} command needs to be executed from within the
directory, where the \LaTeX\ source file is located. In Windows,
the command line can be opened in a directory by holding down the
\keys{Shift} key and by clicking the right mouse button while
hovering the cursor over a directory. Select the \menu{Open Command
Window Here} option in the context menu that opens shortly
afterwards.

With online services -- such as Overleaf -- the commands are
executed automatically, although the locale may be erroneously
detected, or the \texttt{makeindex} tool (which is only able to
sort entries that contain digits and letters of the English
alphabet) may be used instead of \texttt{texindy}. In either case,
the index will be ill-sorted.

\makeatletter\thesis@blocks@clear\makeatother
\phantomsection %% Print the index and insert it into the
\addcontentsline{toc}{chapter}{\indexname} %% table of contents.
\printindex

\makeatletter\thesis@blocks@clear\makeatother
%% Patch the appendix hyperrefs (see:
%% <http://tex.stackexchange.com/q/174887/70941>)
\renewcommand{\theHchapter}{A\arabic{chapter}}
\appendix %% and start the appendices.

\chapter{An appendix}
Here you can insert the appendices of your thesis.

\end{document}
